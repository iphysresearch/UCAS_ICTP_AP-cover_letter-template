\documentclass[a4paper,12pt]{article}
\usepackage{graphicx}
\usepackage{geometry}
\usepackage{fancyhdr}
\usepackage{parskip} % 段落间距
\usepackage{xcolor}  % 用于颜色支持
\usepackage[T1]{fontenc}
\usepackage{lmodern}
\usepackage{newtxtext,newtxmath} % Times New Roman 样式

% 定义自定义颜色
% https://m.fontke.com/tool/cmyk/94,75,10,0
% https://www.colorcodeslab.com/zh/hex/114d93/
\definecolor{MyBlue}{RGB}{17,77,147}  
\definecolor{MyGray}{gray}{0.5}
\definecolor{MyBlack}{gray}{0.1}

% 定义颜色切换命令
% \newcommand{\MyColor}{MyBlack} % 默认使用 MyBlack
\newcommand{\MyColor}{MyBlue} % 切换为 MyBlue

% 设置页面边距
\geometry{
  top=3cm,
  bottom=2cm,
  left=2.5cm,
  right=2.5cm
}

% 设置页眉和页脚
\pagestyle{fancy}
\fancyhf{}
\fancyhead[L]{\includegraphics[width=0.6\textwidth]{logo_ucas.jpg}} % 国科大 logo
% \fancyhead[L]{\includegraphics[width=0.6\textwidth]{logo_ictp-ap.jpg}} % ICTP-AP logo
\fancyhead[R]{\textcolor{\MyColor}{\today}} % [\today] 自动生成当前日期


\begin{document}

\bigskip % Vertical whitespace
\vspace*{1cm} % 使用 \vspace* 强制增加垂直空间

\hfill
\begin{tabular}{l @{}}
% \hfill \today \bigskip\\ % Date
\hfill \textbf{\textcolor{\MyColor}{[Name]}} \\ % [Name] 替换为你的名字
\hfill \textcolor{MyGray}{Email: [your.email@ucas.ac.cn]} \\ % [your.email@ucas.ac.cn] 替换为你的邮箱
\hfill \textcolor{MyGray}{No.55, Zhongguancun East St. Haidian Dist.} \\ % 地址,无需修改
\hfill \textcolor{MyGray}{Beijing, P. R. China, 100190} \\ % 地址,无需修改
\hfill \textcolor{MyGray}{Phone: [(123) 456-7890]} \\ % [(123) 456-7890] 替换为你的电话号码
\end{tabular}


\bigskip % Vertical whitespace

\vspace*{1cm} % 使用 \vspace* 强制增加垂直空间

% ADDRESSEE AND GREETING
\begin{tabular}{@{} l}
	\textbf{\textcolor{\MyColor}{Professor [Chandeller Bing]}} \\ % [Chandeller Bing] 替换为编辑的名字
	\textbf{\textcolor{\MyColor}{Editor-in-chief}} \\ % [Editor-in-chief] 替换为编辑的职位
	\textit{\textcolor{\MyColor}{[Journal of Old Friends]}} % [Journal of Old Friends] 替换为目标期刊名称
\end{tabular}

\bigskip % Vertical whitespace

% 信件正文
\begin{flushleft}
    \textbf{\textcolor{\MyColor}{Dear Editor,}} % [Dear Editor] 替换为适当的称呼
\end{flushleft}

\vspace{0.5cm}

% 信件内容
\noindent
We are pleased to submit our manuscript entitled "\textbf{Title of Your Paper}" for consideration for publication in the APS journal. This work presents [briefly describe the main focus and findings of your research]. We believe that our findings will be of interest to the readers of [specific APS journal name] due to [reason why your work is relevant to the journal].

\vspace{0.5cm}

Our study provides new insights into [mention the specific field or topic], and we have employed [mention any specific methodologies or approaches] to ensure the robustness of our results. We believe that our work contributes significantly to the ongoing research in this area and opens up new avenues for further investigation.

\vspace{0.5cm}

We confirm that this manuscript has not been published elsewhere and is not under consideration by any other publication. All authors have approved the manuscript and agree with its submission to [specific APS journal name].

\vspace{0.5cm}

Thank you for considering our submission. We look forward to your positive response.

\vspace{1cm}

% 作者信息
\begin{flushright}
    \begin{minipage}{1.0\textwidth} % 调整宽度以适应布局
        \raggedright
        \textbf{\textcolor{\MyColor}{Sincerely,}} \\
        \vspace{0.5cm}
        \textbf{\textcolor{\MyColor}{[Your Name]}}[, Ph.D. candidate] \\ % [Your Name] 替换为你的名字,[, Ph.D. candidate] 可选,根据你的身份修改
        International Centre for Theoretical Physics Asia-Pacific \\
        University of Chinese Academy of Sciences   \\
    \end{minipage}
\end{flushright}

\end{document}